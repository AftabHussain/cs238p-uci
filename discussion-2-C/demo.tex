\documentclass[10pt]{beamer}

\usetheme[progressbar=frametitle]{metropolis}
\usepackage{appendixnumberbeamer}

\usepackage{booktabs}
\usepackage[scale=2]{ccicons}

\usepackage{pgfplots}
\usepgfplotslibrary{dateplot}

\usepackage{xspace}
\usepackage{minted}
%\usepackage{tabularx}
\usepackage{bytefield}
\newcommand{\themename}{\textbf{\textsc{metropolis}}\xspace}
\setminted[]{autogobble, breaklines, breakafter=-/,fontsize=\footnotesize}
\title{C - Basic Features \& Pointers} %Introduction to 
\subtitle{CS238P: Principles of operating systems - Fall'18}
% \date{\today}
\date{}
%\author{Matthias Vogelgesang}
\institute{University of California, Irvine}
% \titlegraphic{\hfill\includegraphics[height=1.5cm]{logo.pdf}}

\begin{document}

\maketitle

%\section{Basics}

\iffalse
\begin{frame}{Data types}
\begin{itemize}
\item char (1 byte)
\item int, long (4/8 bytes)
\item float, double
\end{itemize}
\end{frame}

\begin{frame}{Loops}
\begin{itemize}
\item \texttt{for}
\item \texttt{do...while}
\item \texttt{while}
\end{itemize}
\end{frame}
\fi

\begin{frame}[fragile]{Hw1(xv6 shell)}
\begin{itemize}
\item<1-> \texttt{if...else}
\begin{minted}{c}
  pid = fork();
  if(pid == -1)
    perror("fork:");
\end{minted}
\item<2-> \texttt{switch...case}
\begin{minted}{c}
  switch(cmd->type){
  case '>': ...; break;
  default: ...; break;
  }
\end{minted}
\item<3-> Functions
\begin{itemize}
\item Process creation (\texttt{fork, exec})
\item File I/O (\texttt{open, close, read, write})
\begin{minted}{c}
fd = open(rcmd->file, rcmd->mode);
\end{minted}
\end{itemize}
\item<4-> Typecasting (next slide)
\item<5-> Command line arguments (\texttt{argv})
\end{itemize}
\end{frame}

\begin{frame}[fragile]{Typecasting}
\begin{itemize}
\item<1-> Change the type of the object for a single operation
\begin{minted}{c}
  var = (dest_type) source;
\end{minted}
\item<2-> Pass generic objects
\begin{minted}{c}
struct cmd { int type; };
struct execcmd {
  int type;
  char *argv[MAXARGS];
};
void runcmd(struct cmd *cmd) {
    ...
    ecmd = (struct execcmd*)cmd;
}
struct cmd* execcmd(void) {
  struct execcmd *cmd;
  ...
  return (struct cmd*)cmd;
}
\end{minted}
\item<3-> Beware of strings! (demo: str.c)
\end{itemize}
\end{frame}

\begin{frame}[fragile]{Arrays}
\begin{itemize}
\item<1-> Collection of objects of the same data type
\item<2-> Accessed by index (\texttt{0 ... size - 1})
\item<3-> String is an array of characters (demo: string.c)
\item<4-> No reference operator
\begin{minted}{c}
printf("Address of a \%p | \%p\n", a, &a);
>> Address of a 0x7aff07024060 | 0x7aff07024060
\end{minted}
\end{itemize}
\end{frame}

\begin{frame}[fragile]{Array Intialization}
Designated Initializers\footnote{http://gcc.gnu.org/onlinedocs/gcc-4.0.4/gcc/Designated-Inits.html}
\begin{minted}{c}
#define CAPSLOCK (1<<3)
#define NUMLOCK (1<<4)
#define SCROLLLOCK (1<<5)
static uchar togglecode[256] = {
[0x3A] CAPSLOCK,
[0x45] NUMLOCK,
[0x46] SCROLLLOCK
};
/* equivalent to */
togglecode[0x3A] = CAPSLOCK;
togglecode[0x45] = NUMLOCK;
togglecode[0x46] = SCROLLLOCK;
\end{minted}
Initialize the array elements 0x3A, 0x45, 0x46 only~\footnote{sheet 77, xv6-rev9.pdf}
\end{frame}

\begin{frame}[fragile=singleslide]{Bit fields\footnote{sheet 09 xv6-rev9.pdf}}
\begin{minted}{c}
// Gate descriptors for interrupts and traps
struct gatedesc {
  uint off_15_0 : 16; // low 16 bits of offset in segment
  uint cs : 16; // code segment selector
  uint args : 5; // # args, 0 for interrupt/trap gates
  uint rsv1 : 3; // reserved(should be zero I guess)
  uint type : 4; // type(STS_{TG,IG32,TG32})
  uint s : 1; // must be 0 (system)
  uint dpl : 2; // descriptor(meaning new) privilege level
  uint p : 1; // Present
  uint off_31_16 : 16; // high bits of offset in segment
};

struct gatedesc d;
d.s = 0; d.args = 0;
\end{minted}
\end{frame}

\begin{frame}[fragile]{Access low-level data}
\begin{bytefield}[endianness=little,bitwidth=1em]{32}
\bitheader[lsb=32]{32,36,37,39,40,43,44,45,46,47,48,63} \\
\bitbox{5}{args} & \bitbox{3}{rsvd} & \bitbox{4}{type} & \bitbox{1}{\textcolor{red}{s}} & \bitbox{2}{dpl} & \bitbox{1}{p} & \bitbox{16}{offset high} \\ [3ex]
\bitheader{0,15,16,31} \\
\bitbox{16}{offset} & \bitbox{16}{segment} \\
\end{bytefield}
\begin{itemize}
\item<2-> Set bit 44 (s) - Or ($|$) it
\begin{minted}{c}
/* on a 64-bit data type */
data = data | (1 << 44);
data |= (1 << 44);
\end{minted}
\item<3-> Clear a bit (s) - And ($\&$) and Not ($\sim$) 
\begin{minted}{c}
/* on a 64-bit data type */
data = data & ~(1 << 44);
data &= ~(1 << 44);
\end{minted}
\end{itemize}
\end{frame}

\begin{frame}[fragile]{Dynamic registration}
\small
\begin{itemize}
\item<1-> Declare a struct to hold function pointers~\footnote{sheet 40 xv6-rev9.pdf}
\begin{minted}[fontsize=\footnotesize]{c}
#define NDEV  10
#define CONSOLE 1
struct devsw {
  int (*read)(struct inode*, char*, int);
  int (*write)(struct inode*, char*, int);
};
struct devsw devsw[NDEV]; /* global data structure */
\end{minted}
\item<2-> Register function pointer~\footnote{sheet 82 xv6-rev9.pdf}
\begin{minted}{c}
int consolewrite(struct inode *ip, char *buf, int n);
int consoleread(struct inode *ip, char *dst, int n);
devsw[CONSOLE].write = consolewrite;
devsw[CONSOLE].read = consoleread;
\end{minted}
\end{itemize}
\end{frame}

\begin{frame}[fragile]{Pointers \& buffer management}
\begin{itemize}
\item<1-> Access raw memory
\begin{minted}{c}
#define KERNBASE 0x80000000
#define P2V(a) (((void *) (a)) + KERNBASE)
uchar *code;
code = P2V(0x7000);
\end{minted}
\item<2-> kalloc, memset, kfree
\begin{minted}{c}
mem = kalloc(); /* allocate a page */
memset(mem, 0, PGSIZE); /* memset */
kfree(mem); /* free it when done */
\end{minted}
\item<3-> memcpy, memmove
\begin{minted}{c}
/* move start to code */
memmove(code, _binary_entryother_start,
      (uint)_binary_entryother_size);
\end{minted}
\end{itemize}
\end{frame}


%\setbeamercolor{palette primary}%{fg=black, bg=yellow}
\begin{frame}[standout]
  Questions?
\end{frame}

\end{document}
